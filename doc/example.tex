\section{Example: Min oscillations in \emph{E. Coli}}
\label{sec:ex}

In this section we describe the general workflow involved in setting up and
simulating a model in URDME using the Comsol and Matlab
interfaces. The major steps are:

\begin{enumerate}
\item \emph{Specify the model}. This involves defining the geometry,
  mesh, initial conditions and chemical reactions of the model. In
  URDME, this will require the generation of three model
  files: a Comsol model file `model.mph' , a Matlab model file
  `model.m' and a reaction propensity ANSI C file `model.c', where we use
  {\it model} as a placeholder for the non-extension part of the
  file-name.
  
  The Comsol model defines the geometric domain of the problem and
  Comsol Multiphysics is used to create a mesh representing the
  spatial discretization of the diffusion equation with Neumann
  boundary conditions and the corresponding inter-voxel diffusion jump
  coefficients. The Matlab model file specifies the chemical reaction
  networks of the problem by defining a stoichiometric matrix
  \varN\ and a dependency graph \varG. The propensity functions for
  the chemical reactions are specified in the ANSI C model file.
	
\item \emph{Export the FEM-structure from Comsol to Matlab.} After
  defining the geometry and using Comsol to create a tetrahedral mesh
  of the model, the resulting data structure (FEM) is exported from
  Comsol to the Matlab workspace via the built-in Comsol/Matlab
  coupling.
	
\item \emph{Run the simulation.} The solver is launched from the
  Matlab workspace via the interface routine in `urdme.m'. As input,
  you will have to specify the \texttt{.m} and \texttt{.c} model
  files. Internally, \texttt{urdme} uses \texttt{comsol2urdme} to
  initialize the main structure \texttt{umod}.The function defined in the 
  file `model.m' file then appends data to this structure.
  	
\item \emph{Post-processing.} After a normal termination of the
  solver, a trajectory of the stochastic process will be attached to
  the Comsol file stored in \texttt{umod.comsol}. At this point, you can use all post-processing
  options available in the Comsol interface to visualize the results. If you have
  other needs not covered by the built-in routines, you can implement
  your own post-processing scripts in Matlab. If using Comsol Multiphysics 4.x, you
  might as well save the solution stored in \texttt{umod.comsol} to a file via the command \textit{mphsave}
  and observe0 it externally in the Comsol GUI.
\end{enumerate}
 
To illustrate the above steps in detail, we will reproduce simulations
of the Min system from \cite{MinD}. The geometry will be a model of an
\emph{E.~coli} bacterium. It is rod shaped with length $3.5\mu m$ and
diameter $0.5\mu m$. The reactions and parameters of the model can be
found in Table\ \ref{tab:MinD}.

\begin{table}[htp]
\centering
\begin{tabular}{ll}
  $\mbox{MinD\_c\_atp} \xrightarrow{k_d} \mbox{MinD\_m}$ & 
  $\mbox{MinD\_c\_atp}+\mbox{MinD\_m} \xrightarrow{k_{dD}} \mbox{2MinD\_m} $ \\
  $\mbox{Min\_e+MinD\_m} \xrightarrow{k_{de}} \mbox{MinDE}$ & 
  $\mbox{MinDE} \xrightarrow{k_e} \mbox{MinD\_c\_adp} + \mbox{Min\_e} $ \\
  $\mbox{MinD\_c\_adp} \xrightarrow{k_p} \mbox{MinD\_c\_atp}$
 \end{tabular} 
  \caption{The chemical reactions of the MinD/MinE model. The
    constants take the values $k_d = 0.0125\mu m^{-1}s^{-1}$, $k_{dD}
    = 9 \times 10^6 M^{-1}s^{-1}$, $k_{de} = 5.56\times 10^7
    M^{-1}s^{-1}$, $k_e=0.7s^{-1}$ and $k_p = 0.5s^{-1}$.}
  \label{tab:MinD}
\end{table}

\subsection{Setting up the model for simulation}

The following steps shows how to create the Comsol model file. If you
don't want to go through all the steps yourself, open the example file
`coli.mph' in the `examples/mincde' folder.

\paragraph*{Defining the geometry and diffusion rates in Comsol Multiphysics 3.5}

\begin{enumerate}
\item Open Comsol and select `Chemical engineering module -- Mass
  transport -- Diffusion -- Transient analysis' (3D). In the
  `Dependent variables' field write \texttt{MinD\_c\_atp MinD\_m,
  Min\_e MinDE MinD\_c\_adp}. These are the names of the variables
  that we will use. Select \emph{Lagrange -- Linear} elements and
  press `OK'.
  
    {\bf \textcolor{red}{!}} \emph{Note that the `Chemical engineering module' is not required 
    in general for URDME, but is used in this example for convenience.}


  \item Next we create the geometry. We will build the rod shaped
  domain from two spheres and one cylinder. Press the `Cylinder'
  button and in the radius and height field enter \texttt{0.5e-6} and
  \texttt{3.5e-6} and press `OK'. You should now see a cylinder in
  your workspace. In the `Draw mode' action bar, press `Zoom extents'
  in order to get a better view of the domain.  Press the `sphere'
  button and enter \texttt{0.5e-6} in the radius field and press
  `OK'. Create another identical sphere but enter \texttt{3.5e-6} as
  the $z$-coordinate. Select all three figures and press the `union'
  button and then the `Delete interior boundaries' button.

\item Having defined the species and the geometry, the next step is to
  specify the parameters in the model. In the menu `Physics $>$
  Subdomain settings', choose subdomain 1 and set the diffusion
  constants to \texttt{2.5e-12} for \texttt{MinD\_c\_adp,
    MinD\_c\_atp} and \texttt{Min\_e}. For \texttt{MinDE} and
  \texttt{MinD\_m} the diffusion constant should be \texttt{1e-14}.
  
  \smallskip 
   
  {\bf \textcolor{red}{!}} \emph{\texttt{MinDE} and \texttt{MinD\_m}
    are membrane bound species, hence their lower diffusion rates. We
    have not specified this explicitly at this stage, but will do so
    later in the \texttt{Matlab} model file.}

\item In order to be able to distinguish between the interior of the
  bacterium and the membrane, we must also create two domains. One
  interior domain that represents the cytoplasm and one boundary
  domain that represents the membrane. This is done by defining the
  special variable \varrdmesd\ as an expression with different value
  in the different subdomains. It can then be used to distinguish the
  nodes on the boundary from those in the interior. Select `Options
  $>$ Subdomain expressions' and enter \varrdmesd\ with value 1 and
  click `OK'. Select `Options $>$ Boundary expressions' and select all
  boundaries (there should be 12 of them). Enter \varrdmesd\ with
  value 2. Finally select `Options $>$ Global expressions' and enter
  \varrdmesdlevel\ with value 2 indicating that the \emph{lowest}
  dimension where \varrdmesd\ is defined is on the surface of the
  geometry.

\item In the `Mesh' menu, select `Mesh $>$ Free mesh parameters' and
  choose `Custom mesh size'.  Set the maximum element size to
  \texttt{1e-7} and press `Initialize mesh'.  Now select `Solve $>$
  Update model'.  Make sure that you are connected to Matlab, if not,
  connect via `File $>$ Client/Server/Matlab $>$ Connect to MATLAB'.
  Then export the FEM structure to the Matlab workspace from the `File
  $>$ export' menu.
\end{enumerate}

\paragraph*{Defining the geometry and diffusion rates in Comsol Multiphysics 4.x}

\begin{enumerate}
\item Open Comsol and use the Model Wizard to create the model template.
Select `3D' as space dimension' and add the physics module `Chemical Species Transport / Transport of Diluted Species' 
in the next step. In the `Dependent variables' window chose the `Number of species' to be \texttt{5}
and in the `Concentrations' list enter the names \texttt{MinD\_c\_atp MinD\_m,
  Min\_e MinDE} and  \texttt{MinD\_c\_adp}. These are the names of the variables
  that we will use.  Select the `Time Dependent' study type in the next step of the wizard and click 
  on the flag symbol to create the template.
  
    {\bf \textcolor{red}{!}} \emph{Note that the `Chemical engineering module' is not required 
    in general for URDME, but is used in this example for convenience.}

\item Next we create the geometry. We will build the rod shaped
  domain from two spheres and one cylinder. Right click on `Geometry 1', 
  select the `Cylinder' option and in the radius and height field enter \texttt{0.5e-6} and
  \texttt{3.5e-6}. Click on the `Build Selected' Button and you should now see a cylinder in
  your workspace. 
  Now, select the `Sphere' node from the `Geometry' context-menu and and enter 
  \texttt{0.5e-6} in the radius field. Create another identical sphere but enter \texttt{3.5e-6} as
  the $z$-coordinate.  Click on `Build All' and observe the created domain in the graphics-window.
  
  Right click on `Geometry' again and select `Boolean Operations $>$ Union'. Select all three
  domains and add them to the `Input objects' selection.  Uncheck the `Keep interior boundaries'
  box and complete the geometry creation by pushing the `Build All' button.

\item Having defined the species and the geometry, the next step is to
  specify the parameters in the model. In the physics settings  `Transport of Diluted Species $>$
  Transport Mechanisms', deactivate the flag on `Convection'.
  Next we need to specify the diffusion constants of the species in the `Diffusion' node of the physics menu.
  Enter the diffusion coefficients \texttt{2.5e-12} for \texttt{MinD\_c\_adp,
    MinD\_c\_atp} and \texttt{Min\_e}. For \texttt{MinDE} and
  \texttt{MinD\_m} the diffusion constant should be \texttt{1e-14}. The units of all constants are $m^2/s$.
  
  \smallskip 
   
  {\bf \textcolor{red}{!}} \emph{\texttt{MinDE} and \texttt{MinD\_m}
    are membrane bound species, hence their lower diffusion rates. We
    have not specified this explicitly at this stage, but will do so
    later in the \texttt{Matlab} model file.}

\item In order to be able to distinguish between the interior of the
  bacterium and the membrane, we must also create two domains. One
  interior domain that represents the cytoplasm and one boundary
  domain that represents the membrane. This is done by defining the
  special variable \varrdmesd\ as an expression with different value
  in the different subdomains. It can then be used to distinguish the
  nodes on the boundary from those in the interior. 
  
  Click right on the menu `Definitions', and create two `Variables' elements.
  In the first one, select the `Geometry entity level' to be `Domain' and chose
  the `Selection' to be `All domains'.  Now, enter a new variable in the window below
  by specifying the name to \texttt{rdme\_sd} and expression to \texttt{1}.
  
  In the second Variable-element, specify the geometric entity level to `Boundary'
  and set the `Selection' to `All boundaries'. Enter the variable name \texttt{rdme\_sd} into
  the `Variables' window and set the expression to \texttt{2} indicating that the \emph{lowest}
  dimension where \varrdmesd\ is defined is on the surface of the
  geometry.

\item In the `Mesh' node, set `User controlled mesh' as sequence type and
  in the appeared `Size' node select the `Custom' option.  Set the maximum element size to
  \texttt{1e-7} and press `Build All'.  Now click on the `Study' node and press the `Compute' button.  
  
  Now you need to transfer the created geometry into Matlab. Make sure that you are connected to the Server, if not,
  connect via `File $>$ Client Server $>$ Connect to Server'.
  When having a working connection the export can be performed by selecting `File $>$ Client Server $>$ Export Model to Server'.
  
  Another option is to save the model as a \textit{mph-file}, and open it later in a  running Matlab
  session with `LiveLink' via the command \texttt{mphload}.
\end{enumerate}

\paragraph*{Specifying the chemical reactions}

The chemical reactions are specified in a separate model file written
in C. This file will be given as input to URDME, which will compile
and launch a solver. Every time the reaction propensity file is
changed, the solver needs to be recompiled, but this will be
automatically detected by URDME.  The way the reaction propensity
functions are specified are explained in more detail in Section
\ref{sec:prop}, which we recommend that you read before continuing
with this example.

The following code specifies the reaction propensity model \texttt{.c}
file for the reactions in Table \ref{tab:MinD}. This file is located
in `examples/mincde/fange.c' in the URDME installation
directory. Either open that file, or create a new one of your own,
entering the code below.

%/* Ordering of the species. */
%const enum {MinD_c_atp = 0,
%            MinD_m     = 1,
%            MinD_e     = 2,
%            MinDE      = 3,
%            MinD_c_adp = 4};

%/* Indicator values of sd. */
%const enum {CYTOSOL = 1, MEMBRANE = 2};

%/* Number of reactions. */
%const int NR = 5;

\begin{verbatim}

#include <stdlib.h>
#include "propensities.h"

/* Ordering of the species. */
#define MinD_c_atp 0
#define MinD_m     1
#define MinD_e     2
#define MinDE      3
#define MinD_c_adp 4

/* Indicator values of sd. */
#define CYTOSOL  1
#define MEMBRANE 2

/* Number of reactions. */
#define NR       5

/* Rate constants. */
const double NA    = 6.022e23;
const double kd    = 1.25e-8;
const double kdd   = 9.0e6; 
const double kde   = 5.56e7;
const double ke    = 0.7;
const double k_adp = 1.0;

/* Reaction propensities. */
double rFun1(const int *x, double t, double vol, const double *data, int sd)
/* MinD_c_atp -> MinD_m */
{
  if (sd == MEMBRANE) 
    return kd*x[MinD_c_atp]/data[0];
  return 0.0;
}

double rFun2(const int *x, double t, double vol, const double *data, int sd)
/* MinD_c_atp + MinD_m -> 2MinD_m */
{
  return kdd*x[MinD_c_atp]*x[MinD_m]/(1000.0*NA*vol);
}

double rFun3(const int *x, double t, double vol, const double *data, int sd)
/* MinD_m + Min_e -> MinDE */
{
  return kde*x[MinD_m]*x[MinD_e]/(1000.0*NA*vol);
}

double rFun4(const int *x, double t, double vol, const double *data, int sd)
{
  return ke*x[MinDE];
}

double rFun5(const int *x, double t, double vol, const double *data, int sd)
/* MinD_c_adp -> MinD_c_atp */
{
   return k_adp*x[MinD_c_adp];
}


PropensityFun *ALLOC_propensities(void)
{
  PropensityFun *ptr = malloc(sizeof(PropensityFun)*NR);
  
  ptr[0] = rFun1;
  ptr[1] = rFun2;
  ptr[2] = rFun3;
  ptr[3] = rFun4;
  ptr[4] = rFun5;

  return ptr;
}

void FREE_propensities(PropensityFun* ptr)
{
  free(ptr);
}
\end{verbatim}

\noindent
There are a few points that deserves highlighting:

\begin{itemize}
\item Note the unit conversions given explicitly in the bimolecular
  propensity function. The rate constants for the bimolecular
  reactions in this model are given in the unit $M^{-1}s^{-1}$ and
  need to be converted to mesoscopic rates. That is why we divide with
  Avogadros number times the volume of the subvolume. Also, the way we
  have set up the geometry model file, the volume is given in the unit
  $m^3$, and needs to be converted to $L^3$. URDME cannot keep track
  of matching the units between the different model files
  automatically: this is the responsibility of the end-user.

\item Note how we use the input \texttt{sd} in the first reaction to
  make sure that it only occurs in subvolumes lying on the
  membrane. We have to make sure, however, that we keep track of what
  value we assigned to the different subdomains in the Comsol model
  file (the value of the expression \texttt{rdme\_sd}).

\item The input \texttt{t} passes the current time to the propensity
  function. This input is included in the typedef of the propensity
  function to make it more general and accommodate future
  needs. However, the NSM-solver does not currently handle time dependent
  propensities correctly. 
  %While an explicitly time dependent  propensity will not generate any run-time errors, 
  Thus, the resulting
  stochastic trajectory will not be a statistically correct
  realization of the intended process.

\item The first reaction in the model contains a scaling with the
  local length scale of the subvolume. For a uniform Cartesian mesh
  this would simply have been the (constant) side lengths of the cubes
  in the mesh. For the unstructured mesh however, this value will be
  different in every subvolume. It is readily obtained from Comsol,
  and is passed to the propensity function via the data vector
  \vardata. \vardata\ will be initialized with the correct values in
  the Matlab model file \texttt{fange.m}, next to be described.
\end{itemize}

\paragraph*{Creating a .m model file}

Before we can run the simulation, we have yet to specify a few more
data structures. We will also need to modify the diffusion rates that
we obtain from the initial Comsol model so that the membrane-bound
species only diffuse on the membrane. We have already prepared for
this by labeling the subvolumes next to the boundary using the
expression \texttt{rdme\_sd} in the Comsol model file `coli.mph'.

Open the file `examples/mincde/fange.m'. We will walk through the
contents of this file and explain what the different parts
do. Additional information can also be found in the comments in the
file.

\begin{enumerate}

\item \emph{The stoichiometric matrix.} To execute the reactions, the
  solvers need to know the stoichiometry of the reactions. This is
  specified via a sparse matrix \varN\ of dimensions
  \varMspecies\ $\times$ \varMreactions. Entry $(i,j)$ in \varN\ tells
  how species $i$ changes upon execution of reaction $j$. The
  following lines of code will set up the stoichiometric matrix for
  our example:

\begin{verbatim}
% Stoichiometric matrix. Every column corresponds to a reaction.
umod.N = sparse([-1  -1  0  0  1 ;...
                  1   1 -1  0  0 ;...
                  0   0 -1  1  0 ;...
                  0   0  1 -1  0 ;...
                  0   0  0  1 -1]);
                
\end{verbatim} 

\item \emph{The dependency graph.} Efficient implementations of
  simulators for large systems uses a dependency graph to minimize the
  re-computation of rates. URDME requires that such a graph \varG, in
  the form of a sparse matrix, be submitted to the NSM-solver. It
  should have dimensions \varMreactions\ $\times$
  (\varMspecies+\varMreactions). The following code sets up \varG\ for
  this example.

\begin{verbatim}
% Dependency graph. The first Mspecies columns indicate the propensities
% that need to be updated when the corresponding species diffuses. The 
% next Mreactions columns work analogously for reaction events.
umod.G = sparse([1 0 0 0 0 1 1 0 0 1;...
                 1 1 0 0 0 1 1 1 0 1;...
                 0 1 1 0 0 1 1 1 1 0;...
                 0 0 0 1 0 0 0 1 1 0;...
                 0 0 0 0 1 0 0 0 1 1;]);
\end{verbatim}

A non-zero entry at row $i$ in column $j$ means that propensity number
$i$ must be updated if species $j$ diffuses ($j \leq $ \varMspecies)
or if reaction $j-$\varMspecies\ occurs ($j > $ \varMspecies).

{\color{red}{{\bf !}}} \emph{A common reason for errors when
  developing a new model is errors in \varN\ or \varG}. A quick way of
setting up the dependency graph is \texttt{umod.G =
  sparse(ones(Mreactions, Mspecies+Mreactions))}. This will make all
propensities be recomputed after each event. While making the code run
slower, this is guaranteed to be correct and can be useful when
debugging your model file.

\item \emph{The initial condition.} There is complete freedom in
  specifying the initial condition. In the present case we simply
  distribute 4002 $MinD\_c\_{atp}$ and 1040 $MinE$ molecules in some
  random way in the entire bacterium.

%\begin{verbatim}
%% Specify the total number of molecules of the species. 
%nMinD = 4002;
%nMinE = 1040;

%u0 = zeros(Mspecies,Ncells);

%% random index to nMinD subvolumes
%ind = floor(NCells*rand(1,nMinD))+1;

%% use sparse to accumulate
%u0(5,:) = full(sparse(1,ind,1,1,Ncells));

%% analogously
%ind = floor(Ncells*rand(1,nMinE))+1;
%u0(3,:) = full(sparse(1,ind,1,1,Ncells));

%fem.urdme.u0 = u0;
%\end{verbatim}
\begin{verbatim}
% Specify the total number of molecules of the species. 
nMinD = 4002;
nMinE = 1040;

u0 = zeros(Mspecies,Ncells);

ind = floor(Ncells*rand(1,nMinE))+1;
u0(3,:) = full(sparse(1,ind,1,1,Ncells));

ind = floor(Ncells*rand(1,nMinD))+1;
u0(5,:) = full(sparse(1,ind,1,1,Ncells));

umod.u0 = u0;
\end{verbatim}

  Note that the code above does not produce a uniformly random initial
  distribution since the volume of each voxel is not taken into
  account.

\item \emph{Specifying the times to output the state of the system.}
  URDME will look for a vector \texttt{tspan} to determine when to
  output the state of the trajectory (the number of events generated
  in a typical realization often exceeds $10^9$ so we can't output
  after each event). Here, we want to sample the system on the time
  interval $[0,200s]$, with output each second. This is achieved by:

\begin{verbatim}
umod.tspan = 0:200. 
\end{verbatim}
 
 \item \emph{Membrane diffusion}. In order to make $MinD\_m$ and
   $MinDE$ diffuse only on the membrane, we will zero out all elements
   in the diffusion matrix that are in the cytosol. To obtain indices
   of those subvolumes we use the information in the \emph{subdomain
     vector} \texttt{sd}. \texttt{sd} will be generated by the
   \varrs\ interface upon calling the solver interface, and will
   contain the information encoded in the expression \texttt{rdme\_sd}
   in the Comsol model file. For more details, see Section
   \ref{sec:details}.

\begin{verbatim}
pm  = find(umod.sd == 2);
cyt = find(umod.sd == 1);
\end{verbatim} 
 
  Remember that we gave \texttt{rdme\_sd} the value 2 on the membrane
  and 1 in the interior. The diffusion matrix \texttt{D} will contain
  the rate constants for the diffusive events on the unstructured
  mesh. \texttt{D} is also generated from the Comsol model file when
  calling the solver, and will be available to the .m model file in
  \texttt{umod.D}. To (efficiently) zero out the correct entries
  in \texttt{D}, we first decompose the sparse matrix, find the
  entries using \texttt{pm} and \texttt{cyt} above, and then
  reassemble the matrix again (compensating for the removed entries by
  adjusting the diagonal of the matrix).  All in all, the code to do
  this is as follows:

\begin{verbatim}
% For MinD_m (2) and MinDE (4), flag all dofs in the cytosol for 
% removal. 
ixremove = [];
for s = [2 4]
  ixremove = [ixremove; Mspecies*(cyt-1)+s];
end

D = umod.D';

% Decompose the sparse matrix. 
[i,j,s] = find(D);

% Set all elements in the diffusion matrix corresponding 
% to the cytosol to zero.
ixremove = [find(ismember(i,ixremove)); find(ismember(j,ixremove))];
i(ixremove) = [];
j(ixremove) = [];
s(ixremove) = [];

% Reassemble the sparse matrix and adjust the diagonal entries. 
ixkeep = find(s > 0);
D = sparse(i(ixkeep),j(ixkeep),s(ixkeep),Ndofs,Ndofs);
d = full(sum(D,2));
D = D+sparse(1:Ndofs,1:Ndofs,-d);

umod.D = D';
\end{verbatim}

 {\color{red}{\bf !}} \emph{It is of fundamental importance that the
   columns of \texttt{D} sum to zero, and that all off-diagonal
   entries are positive. For an introduction to how \texttt{D} is
   constructed, see Appendix \ref{app:coeffs}. For a detailed account,
   consult \cite{SPDEPEFHL}}.
 
 {\color{red}{\bf !}} \emph{The way we have modeled membrane diffusion
   is simply by saying that the subvolumes closest to the membrane
   constitute the membrane layer. As the mesh becomes finer near the
   boundary, the thickness of this layer will decrease, eventually
   approaching a 2D model of the membrane. One can think of other ways
   of modeling the membrane diffusion. The most obvious is to
   explicitly draw the membrane as a separate (true) subdomain with a
   fixed thickness in the Comsol model file. This would usually mean
   that more subvolumes are needed to resolve that thin layer.}
 
 \item \emph{The generalized data vector.} Finally, we need to set
   \texttt{umod.data} to contain the values of the length
   parameter for the subvolumes (it is needed in the first reaction,
   \texttt{rFun1}). To do this, we use the built-in Comsol function
   \texttt{postinterp} in version 3.5 or \texttt{mphinterp} in version 4.x 
   which can be used to evaluate an expression in
   any point in the domain. Here, we simply get the subvolume sizes by
   using the pre-defined expression \emph{h}, evaluated in the
   vertices of the mesh.

When using Comsol Multiphysics 3.5 we type:
\begin{verbatim}
dofs = xmeshinfo(fem,'Out','dofs');
umod.data = postinterp(umod.comsol,'h',dofs.coords(:,1:Mspecies:end));
umod.data = umod.data(dofs.nodes(1:Mspecies:end));
\end{verbatim}

In Comsol Multiphysics 4.x we use the following code instead:
\begin{verbatim}
xmi = mphxmeshinfo(umod.comsol);
umod.data = mphinterp(umod.comsol,'h','coord', ...
xmi.dofs.coords(:,1:Mspecies:end), 'solnum', 1);
umod.data = umod.data(xmi.dofs.nodes(1:Mspecies:end)+1);
\end{verbatim}

{\color{red}{\bf !}} \emph{The ordering of the vertices in the mesh 
 in the FEM structure (e.g.\ \texttt{fem.mesh.p}) is not necessarily the same as the ordering in 
 Comsol's extended mesh format of the degrees of freedom.  To ensure that the ordering is consistent 
 between these two structures we can alternatively transform them using the following method in Comsol 3.5:}

\begin{verbatim}
dofs = xmeshinfo(fem,'Out','dofs');
data = postinterp(fem,'h',fem.mesh.p); 
umod.data = data(dofs.nodes(1:Mspecies:end));
\end{verbatim}

\emph{For more details concerning the internal ordering of the dofs,
  consult the Comsol user's manual. The interface routines
  \texttt{comsol2urdme} and \texttt{urdme2comsol} also contain useful
  information on this matter. }

%\emph{Note that the subdomain data vector \varsd\ is a completely 
%general piece of information. In particular, it need not be explicitly
%related to the `subdomain' numbers in \Comsol.}

\end{enumerate}

\subsection{Running the simulation}

With all three model files set up correctly, we are now ready to
launch the simulation. In Matlab, change the current working directory
to `examples/mincde' (or if you have prepared your files in another
directory, to that one). 

First, we need to load the Comsol Multiphysics FEM-model into the Matlab
workspace, either by requesting it from the server or by typing:

\begin{verbatim}
>> fem = mphload('coli.mph');
\end{verbatim}

Asuming \texttt{coli.mph} is the file name of the model previously created in Comsol Multiphysics.
To launch the simulation, call the main interface routine in `urdme.m':

\begin{verbatim}
>> umod = urdme(fem,'mincde');
\end{verbatim}

URDME will now extract information from the Comsol and Matlab model
files, compile the solver with linking to the propensities specified in
\texttt{mincde.c}, and then execute the solver by making a system call.

\subsection{Post-processing}

If the simulation in the previous step completed without errors, the
model structure will now contain a realization of the stochastic
process. To visualize the trajectory, we can use any of the
visualization options available in Comsol or we can create routines of
our own. To look at the \texttt{MinD\_m} distribution on the membrane
at the final time we can use Comsol's post-processing functionality:

\paragraph*{Post-processing using Comsol Multiphysics 3.5}

\begin{verbatim}
>> postplot(umod.comsol,'Tetdata','MinD_m');
\end{verbatim} 

To visualize the result at any other time, e.g after 100s:
\begin{verbatim}
>> postplot(umod.comsol,'Tetdata','MinD_m','T',100);
\end{verbatim}

 {\color{red}{\bf {!}}} \emph{You can specify any time in the interval
   you simulated, but if you specify a time that lies between two
   points in \texttt{tspan} Comsol will do interpolation to
   approximate the result at that point.}

\noindent
To visualize a species inside the domain, we can do as follows:

\begin{verbatim}
>> postplot(umod.comsol,'Slicedata','MinD_c_atp');
\end{verbatim} 

\noindent
There are many more options that can be passed to \texttt{postplot} to
control the plot produced. For a detailed account , see the Comsol
documentation:
\begin{verbatim}
>> help postplot
\end{verbatim}

\noindent
If you prefer to work within the Comsol GUI for visualization, you can
import the FEM structure stored in \texttt{umod.comsol} with the attached stochastic trajectory back
into Comsol. 

\noindent
Then, from the Comsol GUI, import the new structure (umod.comsol): 'File $>$
Import $>$ FEM structure...'. You can now visualize the trajectory
using the menu 'Postprocessing $>$ Plot Parameters'.
\paragraph*{Post-processing using Comsol Multiphysics 4.x} 

\begin{verbatim}
>> umod.comsol.result.create('res1','PlotGroup3D');
\end{verbatim} 

This command creates a plot-contanainer for the following visualization. \\

To visualize the result at a specific time, e.g after 100s:

\begin{verbatim}
>> umod.comsol.result('res1').set('t','100');
\end{verbatim}

 {\color{red}{\bf {!}}} \emph{You can specify any time in the interval
   you simulated, but if you specify a time that lies between two
   points in \texttt{tspan} Comsol will do interpolation to
   approximate the result at that point.} \\

\noindent

To visualize the result of the simulation on the surface we can use:

\begin{verbatim}
>> umod.comsol.result('res1').feature.create('surf1', 'Surface');
>> umod.comsol.result('res1').feature('surf1').set('expr', 'MinD_m');
>> mphplot(umod.comsol,'res1');
\end{verbatim}

Where we can replace the string \texttt{'MinD\_m'} with the name of any other occupied species. \\

To visualize the solution inside the domain, we need to first create a new plot container.

\begin{verbatim}
>> umod.comsol.result.create('res2', 'PlotGroup3D');
>> umod.comsol.result('res2').set('t','100');
\end{verbatim} 

Now we can visualize the solution on a `slice' of the $zx$-axis of the model.

\begin{verbatim}
>> umod.comsol.result('res2').feature.create('slc2', 'Slice');
>> umod.comsol.result('res2').feature('slc2').set('expr', 'MinD_c_atp');
>> umod.comsol.result('res2').feature('slc2').set('quickplane', 'zx');
>> umod.comsol.result('res2').feature('slc2').set('quickynumber', '1');
>> mphplot(model,'res2');
\end{verbatim} 

\noindent
There are many more options that can be passed to \texttt{postplot} to
control the plot produced. For a detailed account, see the Comsol
documentation:
\begin{verbatim}
>> help mphplot
\end{verbatim}

\noindent
If you prefer to work within the Comsol GUI for visualization, you can
import the FEM structured with the attached stochastic trajectory back
into Comsol. This can be done by typing:

\begin{verbatim}
>> mphsave(umod.comsol,'output_filename.mph')
\end{verbatim}

\noindent
Optionally, from the Comsol GUI, import the new structure (umod.comsol): 'File $>$
Client Server  $>$ Import Model from Server'. You can now visualize the trajectory
using the options provided in the `Results' node.
