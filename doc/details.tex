\section{Details and specifications}
\label{sec:details}

In this section we give a detailed description of the input to \varrs\
and explain how the coupling between the Comsol/Matlab interface and
the solvers works.

\subsection{Input to the solver}
\label{sec:speci}

Table~\ref{tab:rdmeinput1} summarizes the input to the NSM solver. For
precise type definitions, consult the preamble of the source file
`nsmcore.c'. While specific for the NSM solver, most of the input data
are likely to be needed by any simulation algorithm. Furthermore,
contributed solvers should preferably accept the same set of input as
the NSM solver for compatibility across models.

\begin{table}
  \begin{small}
\begin{tabular}{|l|p{0.3\linewidth}|p{0.45\linewidth}|}
  \hline
  {\bf{Name}} & {\bf{Type}} & {\bf{Description}}			\\
  \hline
    &&									\\
  \varNcells & scalar (int) & Number of subvolumes.			\\
  \varMspecies & scalar (int) & Number of different species. This 
  also defines \varNdofs $:=$ \varMspecies$\times$\varNcells.		\\
  \varMreactions & scalar (int) & Number of reactions.			\\
  \vardsize & scalar (int) & Size of the data vector used in the 
  propensity function.							\\
    &&									\\
 \varu & Matrix[\varMspecies$\times$\varNcells] (int) & \varu$(i,j)$ 
  gives the initial number of species $i$ in subvolume $j$. 		\\
  \vartspan &  vector (double) & An increasing sequence of points in 
  time where the state of the system is to be returned.			\\
  \varprop & Vector[\varMreactions] (\texttt{PropensityFun}) 
    & Propensity function pointers. See Section~\ref{sec:prop} for 
  details.								\\
  \varreport & \texttt{ReportFun} & Pointer to a report function. This 
  function is called every time the chain reaches a value in \vartspan. \\
    &&									\\
  \varvol & Vector[\varNcells] (double) 
    & The volume of the macroelements, i.e. the diagonal elements of 
    the lumped mass-matrix $M$ (cf.~Appendix~\ref{subsec:mesodiff}). \\
  \varsd &  Vector[\varNcells] (int) 
    & The subdomain numbers of all subvolumes. See Section~\ref{sec:ex} 
    for more details.							\\
  \vardata &  Matrix[\vardsize$\times$\varNcells] (double) 
    & Generalized data vector. A pointer to column $j$ is passed as an 
    additional argument to the propensities in subvolume $j$.		\\
    &&									\\
  \varD & Sparse matrix[\varNdofs $\times$ \varNdofs] (double)
    & The \emph{transpose} of the diffusion matrix $M^{-1}K$ obtained 
    from the FEM discretization of the macroscopic diffusion equation, 
    cf.~\eqref{eq:FEM}. Each column in \varD\ (i.e. each row in 
    $M^{-1}K$) corresponds to a subvolume, and the non-zero coefficients 
    $D(i,j)$ give the diffusion rate constant from subvolume $i$ to 
    subvolume $j$. 							\\
  \varN & Sparse matrix[\varMspecies\ $\times$ \varMreactions] (int) 
    & The stoichiometric matrix. Each column corresponds to a reaction, 
    and execution of reaction $j$ amounts to adding the 
    $j$th column to the state vector.					\\
  \varG & Sparse matrix[\varMreactions\ $\times$ 
    (\varMspecies+\varMreactions)] (int) 
    & Dependency graph. The first \varMspecies\ columns correspond to 
    diffusion events and the following \varMreactions\ columns to 
    reactions. A non-zeros entry in element $i$ of column $j$ indicates 
    that propensity $i$ needs to be updated if the event $j$ 
    occurs. See Section~\ref{sec:ex} for examples.			\\
 
  \hline
\end{tabular}
\caption{Input arguments to \varrs. For more details, see the source
  file `nsmcore.c'. All data in the table will be passed to the core
  simulation routine via a C-struct \texttt{urdme\_model} except
  \texttt{prop} and \texttt{report} that are specified in separate C
  files. For all sparse matrices, the compressed column sparse (CCS)
  format is used. This is the same format Matlab uses and online
  documentation is available.}
  \label{tab:rdmeinput1}
  \end{small}
\end{table}


\subsection{Specifying propensities for chemical reactions}
\label{sec:prop}

We have provided two separate methods to specify the reaction
propensities. Simple polynomial rate laws (mass-action) can be provided as inline
propensities and can be specified in the Matlab model file. For general propensities and full flexibility, the rate laws can be specified in a model file written in C.  

%The other option is to specify the rate laws in a model file written
%in C. To simplify this process, we

%Note that one can easily use \emph{both} inline and compiled
%propensities simultaneously. This is convenient when only a few
%propensities are complicated and has to be compiled.

%\subsubsection{Inline propensities}
%\label{sec:inline}

An ``inline propensity'' is a compact data format for specifying basic
chemical reactions with polynomial rate laws. An inline propensity $P$
can be defined as
\begin{align*}
  P(x) &= \left\{
 \begin{array}{rl}
    \frac{k_1x_ix_j}{\Omega}+k_2x_k+k_3\Omega & \text{if } i \ne j,	\\
    \frac{k_1x_i(x_i-1)}{2\Omega}+k_2x_k+k_3\Omega & \text{if } i = j.
  \end{array} \right.
\end{align*}
Here $x$ is the column in $\fatx$ which contains the state of the
subvolume considered and $\Omega$ is the corresponding volume. The
coefficients and indices are specified in matrices \varK\ and \varI\
where \varK$(:,r) = [k_1; \, k_2; \, k_3]$ and \varI$(:,r) = [i; \, j;
\, k]$ are the constants corresponding to the $r$th inline
propensity. The matrix \varS\ is a (possibly empty) sparse matrix such
that \varS$(:,r)$ lists all subdomains in which the $r$th inline
propensity is turned off. \emph{Note that no inline propensities are
active in subdomain zero!} A complete example of the use of inline propensities can be found in the 'annihilation' example that ships with URDME 1.2. 



%\subsubsection{Compiled propensities}
%\label{sec:compiled}

The other (and fully general way) to specify propensity functions are
to supply them to \texttt{urdme} as a model file written in ANSI
C. The precise form of the propensity functions is defined by the data
type \texttt{PropensityFun}, defined in the header `propensities.h'
(found in the `include' directory) as

\begin{verbatim}
typedef double (*PropensityFun)(const int *x, double t, double vol, 
                                const double *data, int sd);
\end{verbatim}

The arguments \varvol, \vardata, and \varsd\ to a
\texttt{PropensityFun} are described in
Table~\ref{tab:rdmeinput1}. Additionally, the input vector \texttt{x}
of length \varMspecies\ is the copy number in a given subvolume, and
\texttt{t} is the absolute time. Note that none of the current solvers
are capable of simulating Markov chains with explicitly time-dependent
propensities. The time argument is included in the typedef at this
stage only to simplify future developments.
 
Below is a commented example of a model file defining a simple
chemical system composed of a single species $X$ undergoing a
dimerization reaction.

\begin{verbatim}
/* Propensity definition of a simple dimerization reaction. */
#include <stdlib.h>
#include <stdio.h>
/* Type definition for propensity functions: */
#include "propensities.h"

/* Rate constant. */
const double k = 1.0e-3;

double rFun1(const int *x, double t, double vol, const double *data, int sd)
/*  X + X -> 0. */
{
  return k*x[0]*(x[0]-1)/vol;
}

PropensityFun *ALLOC_propensities(void)
/* Allocation. */
{
  PropensityFun *ptr = (PropensityFun *)malloc(sizeof(PropensityFun));
  ptr[0] = rFun1;

  return ptr;
}

void FREE_propensities(PropensityFun *ptr)
/* Deallocation. */
{
  free(ptr);
}

\end{verbatim}
\noindent
A model file \emph{must} implement the following routines:
\begin{itemize}
  \item \verb#PropensityFun *ALLOC_propensities(void)#
  \item \verb#void FREE_propensities(PropensityFun *ptr)#
\end{itemize}

\noindent
The first function should allocate and initialize an array of function
pointers to the propensity functions and return a pointer to this
array. This is the function that the solvers will call to access the
rate functions.  The second function should deallocate the pointer
\texttt{ptr}. For further examples, see Section \ref{sec:ex}.
