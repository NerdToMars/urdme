\section{Obtaining and installing URDME 1.2}
\label{sec:install}

\paragraph{Usage:}

\textit{URDME is work in progress. You may use, distribute, and
 modify the code freely under the GNU GPL license version 3. We welcome
  suggestions, comments, and bug-reports. Support questions should be directed 
  to the URDME mailing list, which you can sign up for on} \url{http://www.urdme.org}.


\subsection{System requirements and software dependencies}

\begin{itemize}
\item Linux or Apple OSX operating system.
\item Matlab
  \begin{itemize}
  \item Tested versions: 2007a, 2008a, 2008b, 2009a, 2009b, 2011b, 2012a
  \item Command line interface must be installed
  \end{itemize}
\item Comsol multiphysics 3.5a (with appropriate patches) or Comsol multiphysics 4.x
  \begin{itemize}
  \item Tested versions: 3.5a, 4.0, 4.2, 4.3 (recommended)
  \item Must have Matlab integration components installed
  \end{itemize}
\item GCC (Xcode on Apple computers)
  \begin{itemize}
  \item Executables \texttt{gcc} and \texttt{make} must be in the
    path
  \item Standard libraries must be installed
  \end{itemize}
\item The optional SBML support requires additionally 
  \begin{itemize}
  \item Python runtime libraries 2.6 or higher
  \item SBML library for python (\texttt{libsbml})
  \end{itemize}
\end{itemize}

\subsection{Installation procedure}

\begin{enumerate}
\item Obtain the latest release of URDME from
  \url{https://github.com/URDME/urdme}. Download the file {\bf urdme-1.2.tar.gz}. 
\item Unpack the archive. This can be done by running the command
  ``{\bf tar -zxvf urdme-1.2.tar.gz}'' in a terminal. Often it is
  possible to double click the file icon in the operating system's
  graphical file manager.
\item Run the installation script with `system administrator'
  privileges. In a terminal, change directory to {\bf urdme-1.2}
  created from the downloaded archive. Run the script {\bf install.sh}
  with `root' or system administrator privileges. This can usually be
  done by running the command ``{\bf sudo ./install.sh}''. Follow the
  prompts of the installation script, the default values should be
  sufficient for most users. If the installation script runs with no
  errors, URDME should be installed correctly.
\end{enumerate}

\subsection{Testing of the installation {\it and} Quick start guide}
\label{sec:quickstart}

\subsubsection{Using Comsol Multiphysics 3.5}

\begin{enumerate}
\item Start Comsol and open a model file,
  e.g.~\texttt{urdme-1.2/examples/mincde/coli.mph}.
\item From Comsol, start Matlab: \\ \texttt{File >
  Client/Server/MATLAB > Connect to MATLAB}
\item Initialize the Matlab environment.
  \begin{itemize}
  \item Change Matlab's working directory to the folder for the URDME
    model you wish to simulate. At the Matlab command prompt type \\
    \texttt{>> cd urdme-1.2/examples/mincde/}
  \end{itemize}
\item Export the model geometry from Comsol to Matlab.
  \begin{itemize}
  \item Update the Model data:\\ \texttt{Solve >
    Update Model}
  \item Export the data:\\ \texttt{File > Export >
    FEM Structure as 'fem'}
  \end{itemize}
\item Simulate the model. At the Matlab command prompt
  type:\\ \texttt{>> umod = urdme(fem,'mincde')}
\item Visualize the results. At the Matlab command prompt
  type:\\ \texttt{>> postplot(umod.comsol,'Tetdata','MinD\_m')}
\end{enumerate}

\subsubsection{Using Comsol Multiphysics 4.x}

\begin{enumerate}

\item Start the Comsol interface to Matlab (''LiveLink''), \\ \texttt{./comsol server matlab} in Unix-based systems.

  \item Change Matlab's working directory to the folder for the URDME
    model you wish to simulate. At the Matlab command prompt type \\
    \texttt{>> cd urdme-1.2/examples/mincde/}

\item Load the Comsol geometry into Matlab: \\ \texttt{>> fem = mphload('coli.mph')}

\item Simulate the model. At the Matlab command prompt  type:\\ 
  \texttt{>> umod = urdme(fem,'mincde');} 

\item Visualize the results. At the Matlab command prompt type:\\ 
  \texttt{>> umod.comsol.result.create('res1','PlotGroup3D');} \\
  \texttt{>> umod.comsol.result('res1').set('t','900');} \\
  \texttt{>> umod.comsol.result('res1').feature.create('surf1', 'Surface');} \\
  \texttt{>> umod.comsol.result('res1').feature('surf1').set('expr', 'MinD\_m');} \\
  \texttt{>> mphplot(umod.comsol,'res1');} 

\item Optionally, save the output to a \textit{mph}-file for further observations in the Comsol GUI: \\
\texttt{>> mphsave(umod.comsol,'coli\_output.mph')}

\end{enumerate}

\subsubsection{Further possibilities}

To manipulate the model, edit the following files:
  \begin{description}
  \item[\texttt{coli.mph}] (Comsol model geometry, edited via the
    Comsol interface)
    \begin{itemize}
    \item[-] Physical geometry and mesh properties
    \item[-] Names of chemical species
    \item[-] Diffusion coefficients
    \end{itemize}
  \item[\texttt{mincde.m}] (Matlab pre-processing script)
    \begin{itemize}
    \item[-] Stoichiometric matrix
    \item[-] Dependency graph
    \item[-] Initial conditions of chemical species
    \item[-] Custom diffusion rules (i.e.~support for membrane-only species)
    \end{itemize}
  \item[\texttt{mincde.c}] C-functions defining the propensity (rate)
    of each reaction channel.
  \end{description}

A more detailed description on how to set up and simulate this model
is found in Section~\ref{sec:ex}.
