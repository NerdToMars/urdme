\section{Introduction}
\label{sec:intro}

Stochastic simulation methods are frequently used to study the
behavior of cellular control systems modeled as continuous-time
discrete-space Markov processes (CTMC). Compared to the most
frequently used deterministic model, the reaction rate equations, the
mesoscopic stochastic description can capture effects from intrinsic
noise on the behavior of the networks \cite{McAdams:1999fk,
BARKAILEIBLER, Elowitz:2002fj, THATTAI, Paulsson:2000kx}.

In the discrete mesoscopic model the state of the system is the copy
number of the different chemical species and the reactions are usually
assumed to take place in a well-stirred reaction volume. The chemical
master equation is the governing equation for the probability density,
and for small to medium sized systems it can be solved by direct,
deterministic methods \cite{PaulPhD, MacThesis, Ferm:fr, SPGRIDS1,
EngSpect1, EngSpect2}. For larger models however, exact or approximate
kinetic Monte Carlo methods \cite{SSA, TAULEAP} are frequently used to
generate realizations of the stochastic process. Many different hybrid
and multiscale methods have also emerged that deal with the typical
stiffness of biochemical reactions networks in different ways, see
\cite{MSSA, nSSA, MYHYB, Rao:2003df, haseltine:6959, minlic} for examples.

Many processes inside the living cell can not be expected to be
explained in a well-stirred context. The natural macroscopic model is
the reaction-diffusion equation which has the same limitations as the
reaction rate equations. By discretizing space with small subvolumes
it is possible to model the reaction-diffusion process by a CTMC
within the same formalism as for the well-stirred case. A diffusion
event is now modeled as a first order reaction from a subvolume to an
adjacent one and the state of the system is the number of molecules of
each species in each subvolume. The corresponding master equation is
called the reaction-diffusion master equation (RDME) and due to the
very high dimensionality it cannot be solved by deterministic methods
for realistic problem sizes.

The RDME has been used to study biochemical systems in \cite{BISTAB,
MinD}. Here the next subvolume method (NSM) \cite{BISTAB}, an
extension of Gibson and Bruck's next reaction method (NRM) \cite{NRM},
was suggested as an efficient method for realizing sample
trajectories. An implementation on a structured Cartesian grid is
freely available in the software MesoRD~\cite{mesoRD}.

The method was extended to unstructured meshes in \cite{SPDEPEFHL} by
making connections to the finite element method (FEM). This has
several advantages, the most notable one being the ability to handle
complicated geometries in a flexible way. This is particularly
important in cell-biological models where internal structures often
must be taken into account.

This manual describes the software URDME which provides an efficient,
modular implementation capable of stochastic simulations on
unstructured meshes. URDME is easy to use for simulating and studying
a particular model in an applied context, but also for developing and
testing new approximate methods. We achieve this by relying on
third-party software for the geometry definition, meshing,
preprocessing and visualization, while using a highly efficient
computational core written in ANSI C for the actual stochastic
simulation. This keeps the implementation of the Monte Carlo
part small and easily expandable, while the user benefits from
the advanced pre- and post-processing capabilities of modern FEM
software. In this version of URDME, we have chosen to provide an
interface to Comsol Multiphysics 3.5a and 4.x \cite{COMSOL}.
        
The rest of this manual is organized as
follows. Section~\ref{sec:changes} summarizes the major changes to
URDME compared to the earlier version 1.1. Section~\ref{sec:install}
describes the software requirements and the installation procedure. An
overview of the code structure is presented in Section~\ref{sec:ov2}
and the details concerning the input to the code, the provided
interface to Comsol and the way models should be specified are found
in Section~\ref{sec:details}. A URDME model is set up and simulated in
a step-by-step manner in Section~\ref{sec:ex} and in
Section~\ref{sec:integration} we show how a new core solver can be
integrated in the URDME infrastructure.

In two appendices we recapitulate the mesoscopic reaction-diffusion
model and show how the stochastic diffusion intensities are obtained
from a FEM discretization of the diffusion equation. We also list for
reference a few stochastic simulation algorithms.

We refer the interested reader to the full paper \cite{URDMEpaper} for
further information on the URDME software, including comparisons to
other available software and examples of some more advanced usage.
