\section{Code overview}
\label{sec:ov2}

URMDE consists of three logical layers. The top layer is made up of an
interface to an external mesh generator and pre-/post-processing
engine, currently Comsol Multiphysics. The bottom layer is a set
of simulation routines, or solvers, written in ANSI C. Interfacing
those two levels is a middle layer implemented in the Matlab
environment, designed to facilitate data processing and visualization,
as well as custom model development. Together these layers form a
software package that enables the development and efficient simulation
of complex and powerful models of spatial stochastic phenomena.

The URDME structure is designed with both efficiency and flexibility
in mind.  A model is defined by three separate files, one for each of
the logical layers. The geometry of the model is defined in a Comsol
\texttt{.mph} file, along with the names and diffusion rates of each
chemical species. A Matlab model file supplies the model with the
stoichiometric matrix, the dependency graph, and the initial state of
the system. The stoichiometric matrix defines the effect of the
chemical reactions on the state of the system while the dependency
graph indicates the reaction rates that need to be updated after a
given reaction or diffusion event has occurred. Finally, a model file
written in ANSI C specifies the propensity functions for the chemical
reactions in the system. Using compiled rather than interpreted
reaction rates ensures maximum efficiency when simulating the model.

The computational solver that ships with URDME is an efficient
implementation of NSM \cite{BISTAB}. Table~\ref{tab:code} shows the
directory structure of URDME together with a short description of each
routine.

A solver has (at least) two arguments: the path to an input file
in \texttt{.mat} format and a name of an output file on which to store
the trajectory that is generated. The input file will contain all the
data structures the solver needs, each with its specific name. The
URDME C-core contains utility routines to extract this data from the
input file.

The main steps involved in launching a solver is outlined below, along
with the routines that perform the different tasks. Generally, the
user does not have to know or perform all these steps manually: they
are all wrapped in the main routine \texttt{urdme} found in the file
`urdme.m'. However, users planning to write a plug-in solver for URDME
will benefit from a more detailed knowledge of the code structure. The
relevant steps are:

\begin{enumerate}
\item \emph{Process the \texttt{.mph} model file.}  This is achieved
  by exporting the FEM-structure from Comsol to Matlab and invoking
  the routine \texttt{comsol2urdme}. This will initialize a new structure,
  \texttt{umod} storing the original FEM-structure in the field \texttt{umod.comsol}. The
  \texttt{umod} structure contains as well the fields \varD, \varvol, \varsd, i.e.~those
  data structures related to the geometry of the model and to the
  unstructured mesh. Compare Table~\ref{tab:code}.

\item The next step is to invoke the Matlab \texttt{.m} model file to
  initialize the remaining, essential data structures, \varN, \varG,
  \varu, \vartspan\ and optionally \vardata. They should all be added
  as fields to the \texttt{umod} struct. Any modifications of the
  data structures added to the model by \texttt{comsol2urdme} in the
  previous step is typically performed in this step as well.

\item After all required fields in the \texttt{umod} struct have
  been initialized, \texttt{urdme\_validate} is invoked to perform
  error-checking on the input to make sure that all required fields in
  \texttt{umod} are present and has the correct properties.
 
\item Next, all fields in \texttt{umod} is serialized to a
  \texttt{.mat} input-file using \texttt{urdme2mat}.

\item After specifying the propensities in a ANSI C source file, the
  solver(s) are compiled using \texttt{urdme\_compile} and then
  launched by a system call from within the Matlab environment (or
  directly from e.g. a bash terminal).

\item The input file is now read by the solver, using the utility
  routine \texttt{read\_model} found in
  `matmodel.h'. \texttt{read\_model} allocates, initializes and
  returns a C-struct \texttt{urdme\_model}. This \texttt{urdme\_model}
  struct is then the sole input to the routine \texttt{nsm} found in
  `nsm.c'. \texttt{nsm} unpacks the structure and calls the main
  simulation routine \texttt{nsm\_core} found in 'nsmcore.c'. A
  similar construction should be used by contributed solvers for easy
  integration, see Section \ref{sec:integration}.

\item After successful simulation, the resulting trajectory is written
  to an output file in \texttt{.mat} format. This file is then loaded
  back into Matlab and added to the FEM-struct by
  \texttt{urdme2comsol} for visualization using Comsol routines or
  custom Matlab scripts.
\end{enumerate} 

\begin{table}[htp]
  \centering
  \begin{tabular}{|l|l|p{0.55\linewidth}|}
    \hline
    {\bf{Directory}} & {\bf{File(s)}} & {\bf{Description}} \\
    \hline
    & &\\
    bin & urdme\_init & Environmental variable helper program.\\
    & &\\
    build & Makefile & GNU Make input file for automatic solver compilation.\\
    & Makefile.nsm & Makefile for the NSM solver.\\
    & &\\
    comsol & comsol2urdme.m 
    & Matlab-script converting Comsol's FEM-struct to a valid \varrs-input. \\
    & urdme2comsol.m & Matlab-script for conversion of the output of 
    \varrs~to the solution format used by Comsol. \\
    & & \\
    doc & manual.pdf & The most recent version of this manual. \\
    & & \\
    include & matmodel.h & Functions to serialize data to/from solvers. \\
      &	propensities.h & Definition of the propensity function datatype.	\\
      & report.h     & Header for report.c. \\
    & & \\
    msrc  & urdme.m & Gateway routine for the solvers. \\
      & urdme\_startup.m & Initializes URDME. \\
      & urdme\_compile.m & Automatic solver compilation. \\
      & urdme\_validate.m & Input validation. \\
      & urdme2mat.m & Serialize model data for input to solvers.\\
    & & \\
    src & report.c &  Report function used by \varrs. \\ 
    & & \\
    src/nsm & nsm.h & Header for the NSM solver.\\
      & nsm.c & NSM solver interface.\\
      & nsmcore.c & NSM solver computational core. \\
      &	binheap.h & Header for binheap.c.\\
      & binheap.c & The binary heap used in the NSM solver.\\
    & & \\
    examples & (various) & Contains files specifying the example
    studied in detail in Section~\ref{sec:ex}.\\
    & & \\
    \hline
  \end{tabular}
  \caption{Overview of the files and routines that make up URDME.}
  \label{tab:code}
\end{table}

