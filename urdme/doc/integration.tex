\section{Integrating solvers with URDME}
\label{sec:integration}

In this section we will describe how to integrate a third party
spatial stochastic solver into URDME using the DFSP \cite{DFSP} plugin
as an example. URDME plugins have three main components: the makefile,
the solver executable, and (optionally) a pre-execution script. Each
part is described in Table~\ref{tab:dfspIntgration} where the files
that make up the DFSP solver are explained. We recommend that
developers follow this format when integrating their own solvers.

\begin{table}[htb!]
  \centering
  \begin{tabular}{|l|l|p{0.55\linewidth}|}
    \hline
    {\bf{Directory}} & {\bf{File(s)}} & {\bf{Description}}\\
    \hline
    & & \\
    build & Makefile.dfsp & Solver makefile, for building the 
    solver automatically when calling \varrs.  The name of this file
    is very important: the automatic compilation process looks for a
    makefile that is suffixed with the name of the solver (in lower
    case).  This makefile compiles the solver with the model's
    propensity functions into the low-level executable which is called
    by the middle-level interface. \\
    & & \\
    src/dfsp & dfsp.c & Solver entry point and data initialization
    code. Contains a \texttt{main}-function and setup routines for
    the data structures. The initialization procedure takes the
    \texttt{.mat} file (which is a serialization of the
    \texttt{umod} structure from the Matlab workspace) and
    instantiates a \texttt{urdme\_model} struct (defined in
    include/matmodel.h). \\
    & dfsp.h & Header file containing all function prototypes
    necessary for DFSP. \\
    & dfspcore.c & Main entry point for the solver: the function 
    \texttt{dfsp\_core}. \\
    & dfsp\_reactions.c & Helper function to process reaction events.\\
    & dfsp\_diffusion.c & Helper function to process diffusion events.\\
    & & \\
    msrc & urdme\_init\_dfsp.m & Pre-execution script to initialize
    data structures before the solver is called.  When executing a
    model with a specified solver, the URDME-interface looks for a
    Matlab-function named \texttt{urdme\_init\_}$<$solver$>$ (that is,
    in the file `urdme\_init\_$<$solver$>$.m'). \\
    \hline
  \end{tabular}
  \caption{Overview of the files that make up the DFSP plugin solver.}
  \label{tab:dfspIntgration}
\end{table}

When the middle level interface calls the solver executable, it passes
all model and geometry data to the solver via a \texttt{.mat} data
file. The names of the input file and the output file are both
specified as command line arguments (i.e.~in the \texttt{argv}
parameter to the \texttt{main} function). The core URDME distribution
includes routines to read and parse this data file into a C-language
struct (see the header file `matmodel.h').  The solver is then called
with the model struct as a parameter.  Once the solver has finished
simulating the model, it attaches the calculated solution trajectory
to the model structure. The solution trajectory is then serialized to
the output file using supplied routines (see `urdme.m').  When the
solver has completed its execution, the middle level interface imports
the serialized solution trajectory and makes it
available to the post-processing and visualization routines. The
logical separation of solvers from the rest of the URDME software
enables streamlined development and debugging of new computational
methods.

For efficient simulation of URDME models, it is necessary to compile
the model specific propensity functions with the routines of the
solver chosen to perform the simulation.  The solver specific
makefiles are responsible for this compilation.  For illustration we
will use the DFSP plugin and the ``mincde'' model as an example
(i.e.~replace `dfsp' with the name of the solver you wish to
integrate). From the middle-level Matlab interface, when the
\texttt{urdme} function is called with the parameter
\texttt{\{'Solver','dfsp'\}}, URDME attempts to compile the executable
`urdme/fange.dfsp' from the propensity function file `fange.c' and
solver files using the compilation specified in `Makefile.dfsp' in
`urdme/build'. The makefile is responsible for all necessary steps of
the compilation process and the target executable is built in the
`urdme' subdirectory of the current working directly, and is named
according to the propensity file (`fange.c') and solver DFSP, thus
`urdme/fange.dfsp'. The automatic compilation process is designed
for ease of use from the middle-level Matlab interface.

Often spatial stochastic simulation methods require additional
processing of geometry and model data before execution can proceed.
In URDME, this is accomplished through the use of a specifically named
Matlab function found in the `urdme/msrc' subdirectory of the URDME
distribution.  For the DFSP plugin, this file is named
`urdme\_init\_dfsp.m'. The function defined in this file must take as
arguments the \texttt{fem} data structure and a variable number of
additional arguments (i.e.~\texttt{varargin}). If \texttt{urdme} is
called with solver arguments, that cell-vector is passed as the second
argument to this function. Implementation of a pre-processing script
provides method developers with a powerful and flexible way to perform
any necessary data transformations for their specific solvers.
