\section{Using the SBML interface}
\label{sec:sbml}
The SBML interface is found in the \texttt{sbml} directory and allows for translation of downloaded SBML files to URDME
compatible model and propensity definitions.

\subsection{Installation procedure}

\begin{enumerate}

\item Install the Python runtime libraries (Version 2.6 or higher)
available at \url{http://www.python.org}.

\item Download and install the official SBML library from \url{http://sbml.org/Software/libSBML}.
Make sure to enable the language interface (API) to Python during the installation.  The detailed guide on
how to compile \textit{libsbml} with the Python API is available in the online documentation of the library.

\end{enumerate}

\subsection{Testing and quick start guide}

\begin{enumerate}

\item Change into the \texttt{sbml} directory of the URDME installation. 

\item You can now use the \texttt{sbml2urdme} translator to generate a propensity and model function
of the \textit{mincde} example, described in detail in chapter \ref{sec:ex}.
Execute the bash-script \texttt{sbml2rdme} in combination with the provided SBML-file \texttt{mincde.xml} as first parameter.

\texttt{./sbml2rdme mincde.xml}

You should obtain the following information:

\texttt{Creating model c-file mincde.c} \\
\texttt{Creating model m-file mincde.m}

\item The script generated a model (.m) and propensity (.c) file with the same filename as the
SBML specification file (in this case 'mincde'). You can proceed with execution the generated files
together with geometries imported from Comsol Multiphysics. \\
\texttt{umod = mphmodel(fem,'mincde')}

Asuming the geometry definition to be available in the variable \texttt{fem}.

  \smallskip 
   
  {\bf \textcolor{red}{!}} The SBML Level 2 specification is not sufficient to describe all properties and dynamics
  of a full URDME model. Although the generated models are fully operational they can be considered
  as drafts which can be manually extended with more specific model requirements. See chapter \ref{sec:ex} for extensions
  of the \textit{mincde}-model.

\end{enumerate}

\subsection{Usage guide}
The SBML translator can be called as a bash-script where \texttt{model.xml} is the SBML definition file and the 
output directory is an optional parameter. \\

\texttt{./sbml2rdme <model.xml> [output directory]} \\

Alternatively, the Python script \texttt{sbml2rdme.py} available in  \texttt{sbml/src} directory of the distribution
can be called with the same parameters within the Python environment.

\subsection{Limitations}
In the actual version the translator supports exclusively SBML Level 2. Reverse reactions are not supported.
